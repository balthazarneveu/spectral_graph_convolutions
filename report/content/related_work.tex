\section{Related work}

\quad In recent years, several works have been published exploring different approaches for disease prediction.

One prevalent approach involved using convolutional neural networks have been proposed in the last few years based only on the imaging fMRI data. 

Dolz et al. \cite{Dolz2018} investigated a 3D CNN for subcortical brain structure segmentation in MRI. The proposed model addresses computational challenges using small kernels and embedding intermediate-layer outputs in the final prediction. Khosla et al.\cite{Khosla2019} evaluated the impact of the chosen brain parcellations on the performance of the 3D CNN model. They proposed an ensemble learning strategy that combines predictions from models trained on connectivity data with different parcellations that improved the classification accuracy. More recently, Sherkatghanad et al. developped a CNN model that used less parameters compared to state-of-the-art models making it less computationally expensive but showed competitive results on autism disorder detection.

Other researchers, for instance \cite{Kong2019}, investigated the use of autoencoders as a way of extracting a small number of features before using a neural networks on these features to preform the classification task for autism disorder detection. ASD-DiagNet proposed by Eslami and al. \cite{Eslami2019} integrated a data augmentation strategy to further improve the robustness of this model.

After Parisot at al. introduced graph convolutional networks as an approach for disease predicition, further works proposed some improvements to this approach. The same authors proposed a better loss function for better generalization, especially for heterogeneous data using metric learning. More recently, Jiang et al. \cite{Jiang2020} proposed a Hierarchical GCN (hi-GCN) model for learning deep representations from fMRI brain connectivity networks. The model has a two-level GCN architecture, capturing region-to-region brain activity correlations and subject-to-subject relationships simultaneously.

These diverse approaches contribute to the evolving landscape of disease prediction using neural networks, offering insights into novel architectures, optimization strategies, and the integration of multi-modal data for improved accuracy and clinical relevance.
