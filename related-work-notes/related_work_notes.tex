\documentclass[sigconf]{acmart}

%%
%% \BibTeX command to typeset BibTeX logo in the docs
\AtBeginDocument{%
  \providecommand\BibTeX{{%
    Bib\TeX}}}



\setcopyright{acmcopyright}
\copyrightyear{2023}
\acmYear{2023}
% \acmDOI{XXXXXXX.XXXXXXX}

%% These commands are for a PROCEEDINGS abstract or paper.
\acmConference[MVA 2023]{MVA 2023}{December
  2023}{Paris, FR}

% \acmPrice{15.00}
% \acmISBN{978-1-4503-XXXX-X/18/06}


\begin{document}

%%
%% TITLE
\title{Review of Spectral Graph Convolutions for Population-based Disease Prediction}

%%
%% AUTHORS
\author{Manal Akhannous}
\affiliation{%
  \institution{ENPC}
  \streetaddress{Cité Descartes, 8 Av Blaise Pascal}
  \city{Champs-sur-Marne}
  \country{France}
}
\email{manal.akhannous@eleves.enpc.fr}

\author{Ines Vati}
\affiliation{%
  \institution{ENPC}
  \streetaddress{Cité Descartes, 8 Av Blaise Pascal 6 et}
  \city{Champs-sur-Marne}
  \country{France}
}
\email{ines.vati@eleves.enpc.fr}

\author{Balthazar Neveu}
\affiliation{%
  \institution{ENS Paris-Saclay}
  \city{Saclay}
  \country{France}
}
\email{balthazar.neveu@ens-paris-saclay.fr}


\renewcommand{\shortauthors}{MVA et al.}

%% ABSTRACT
\begin{abstract}
  We'll review the paper Spectral Graph Convolutions for Population-based Disease Prediction \cite{Parisot17}
\end{abstract}


%%
%% Keywords. The author(s) should pick words that accurately describe
%% the work being presented. Separate the keywords with commas.
\keywords{Graph, Spectral, Convolution, Medical, Disease}


\received{10 December 2023}
\received[revised]{10 December 2023}
\received[accepted]{10 December 2023}

%% TITLE
\maketitle

\section{Introduction}


%% REVIEW MAIN DOCUMENT
\cite{Parisot17}
\section{Context analysis}

\paragraph{} The studied paper presents a new method for disease prediction using the data available.
The paper introduces Graph Convolutional Networks (GCN) as a technique for analyzing brain data in populations by 
integrating imaging and non-imaging data. The disease diagnosis problem is formulated as graph classification problem and proposes ...
The main objective is to exploit the additional information available with the imaging data 
to incorporate similarities between subjects within a graph structure. 
In this representation, each subject is associated with an imaging feature vector and corresponds to a vertex in the graph. 
The edge weights of the graph are obtained from phenotypic data and encode the pairwise similarity between subjects. 
The GCN model is trained on partially labeled graphs to predict the classes of unlabeled nodes based on their features 
and associations with other subjects.

\paragraph{} The proposed method has various applications. 
It can be used for predicting diseases in cases where imaging data is utilized and the influence of phenotype information
is significant. For example, diagnosing autism spectrum disorder (ASD) can be challenging since there is currently 
no medical test available for diagnosis other than developmental screening. Leveraging the available fMRI scans can 
aid in the diagnosis and make early detection easier, ultimately improving the quality of life for patients. 
This method can be generally applied to classify population graphs. 
Resting state functional (rs-fMRI) %explain briefly
It involves combining data from various sources of different types and representing populations as sparse graphs. 
In these graphs, vertices are associated with individual features, including imaging data, while edges encode information
about the links between individuals. 
Integrating different data sources enhances the quality of predictions.

\section{Related work}
Several works have been published more recently proposing other approaches to disease prediction in the case of the task 
of subject classification in populations (e.g. for diagnosis) that further improved the accuracy of the classification on the ABIDE dataset. 

\bibliographystyle{ACM-Reference-Format}
\bibliography{references}

\end{document}
\endinput
%%
%% End of file `sample-sigconf.tex'.
